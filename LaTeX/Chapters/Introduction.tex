\chapter{Introductie}
Voor het ontwerpen van nieuwe, snellere en veiligere zweeftrein is het essentieel om accurate computermodellen te gebruiken die het rijgedrag van de zweeftrein kunnen simuleren. Voor een accuraat computermodel is een model van de relatie tussen de afstand tussen de twee magneten en de onderlinge kracht essentieel \cite{sanagawa2001characteristics}. In dit onderzoek wordt een model dat gebaseerd is op de superpositie van magnetische puntdipolen getest. \\
Het model, afgeleid en beschreven in Pols \cite{Pols2019}, gaat ervan uit dat de totale potentiele energie van de magneten gelijk is aan de som van potentiele energie van magnetische puntdipolen. Met het gegeven dat de kracht gelijk is aan de afgeleide van de potentiele energie naar de afstand, geldt voor de kracht tussen twee magneten:

\begin{equation}
    F_m=\frac{3\mu_0 m^{2}}{2\pi}\cdot  \frac{1}{z^4} =\alpha/z^4 
\end{equation}
				
Hierin is $\mu_0$ de magnetische permeabiliteit, $m$ het magnetische dipoolmoment en $z$ de hart-tot-hart afstand van de twee magneten. Dit model wordt gevalideerd aan de hand van een experimentele opzet waarbij de afstotende en aantrekkende kracht tussen twee magneten en hun onderlinge afstand wordt bepaald. Uit het experiment wordt het remanente veld bepaald:

\begin{equation}
    B_r=\mu_0 \frac{m}{V}	
\end{equation}

Het model is gevalideerd wanneer het remanente veld volgend uit de empirische data overeenkomt met de specificaties van de magneten. Daartoe richten we ons op de volgende twee onderzoeksvragen:\\
\begin{center}
    1	Hoe hangt de afstotende/aantrekkende kracht tussen twee magneten af van hun onderlinge afstand?\\
    2	Is het theoretische model van een superpositie van dipolen voldoende om die relatie te beschrijven?\\
\end{center}
Dit onderzoek is uitgevoerd als onderdeel van het vak Natuurkundig Practicum aan de TU Delft.
